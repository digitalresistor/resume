\documentclass[11pt,letterpaper,sans]{moderncv}        % possible options include font size ('10pt', '11pt' and '12pt'), paper size ('a4paper', 'letterpaper', 'a5paper', 'legalpaper', 'executivepaper' and 'landscape') and font family ('sans' and 'roman')

% moderncv themes
\moderncvstyle{casual}                             % style options are 'casual' (default), 'classic', 'oldstyle' and 'banking'
\moderncvcolor{orange}                               % color options 'blue' (default), 'orange', 'green', 'red', 'purple', 'grey' and 'black'

% character encoding
\usepackage[utf8]{inputenc}                       % if you are not using xelatex ou lualatex, replace by the encoding you are using

% adjust the page margins
\usepackage[scale=0.88]{geometry}
%\setlength{\hintscolumnwidth}{3cm}                % if you want to change the width of the column with the dates
%\setlength{\makecvtitlenamewidth}{10cm}           % for the 'classic' style, if you want to force the width allocated to your name and avoid line breaks. be careful though, the length is normally calculated to avoid any overlap with your personal info; use this at your own typographical risks...

% personal data
\name{Delta}{Regeer}
\phone[mobile]{+1~480~264~0039}
\email{delta@regeer.org}
\homepage{deltaregeer.com}
\social[linkedin]{deltaregeer}
\social[github]{digitalresistor}
\social[signal]{DigitalResistor.58}
%\extrainfo{additional information}                 % optional, remove / comment the line if not wanted

%----------------------------------------------------------------------------------
%            content
%----------------------------------------------------------------------------------
\begin{document}
%\begin{CJK*}{UTF8}{gbsn}                          % to typeset your resume in Chinese using CJK
%-----       resume       ---------------------------------------------------------
\makecvtitle
\vspace*{-5.5em} % Reduce vertical space after header to create more space
\section{}
Delta has the ability to quickly understand data flow and applications, helping solve challenging problems quickly and efficiently. Working across a range of operating systems, with various different technologies Delta has a wealth of knowledge and experience that allows them to get up to speed quickly. With a deep understanding of networking from routing/switching to application level TCP/UDP they can pinpoint the problem and resolve it. Proficient in multiple programming languages, with their favorites being Rust and Python, and having worked on software in C/C++, Golang, Groovy, Swift, PHP and Java

\section{Experience}


\cventry{2011--2023}{Goon: Parties Lead}{DEF CON}{Las Vegas}{NV}{
\begin{itemize}
\item Coordinated with volunteers to run successful nighttime events at \href{https://www.defcon.org}{DEF CON}
\item On-site: lead a small team of volunteers tasked with making sure the party planners have what they need to make their parties a success
\item Work with DEF CON staff to ensure appropriate contracts/paperwork is provided
\item Plan layout and location of parties and work closely with multiple other departments
\end{itemize}
}

\cventry{2018--2020}{Pyramid/DevOps Consultant}{American Public Power Association}{Arlington}{VA (remote)}{
\begin{itemize}
\item Development of a custom user management web application for licensing and authorization using the Pyramid web framework for multiple oauth2 enabled services, as well as building a frontend using Bootstrap
\item Code review and building test suites for code written in Python
\item Database migrations using Alembic and SQLAlchemy deployed on PostgreSQL
\item Introduced SaltStack and automation to deploying of VMs for running docker-compose based deployments
\item Automation around deployment of development/staging/production environments using Github Actions
\item Maintenance of existing systems running on Digital Ocean
\end{itemize}
}

\cventry{2019--2020}{Senior Security Engineer on Offensive Security}{Cruise LLC}{San Francisco}{CA}{
\begin{itemize}
\item Automated and improved internally written Offensive Security software
    \begin{itemize}
    \item Introduced Github Actions to allow for easier cross platform building of offensive tooling
    \item Built new functionality using C++ using Boost and Boost::Beast to increase malware capabilities
    \end{itemize}
\item Built a Slack bot that sent notifications for tickets in JIRA related to the Vulnerability Management program
    \begin{itemize}
    \item Deployed to AWS using Terraform, API gateway to Lambda, with CloudWatch cron triggers
    \item Integration with Vault and AWS KMS for secret storage
    \item Push-button deployment using Terraform Enterprise after building in CircleCI
    \end{itemize}
\item Helped shape decisions around Vulnerability Management and worked with various internal stakeholders to resolve security vulnerabilities that were found
\item Extensively helped troubleshoot issues with work from home setups with my background in Linux/Docker helping users get productive while at home
\item Worked with external parties to disclose vulnerabilities in their software
\end{itemize}
}

\cventry{2016--2019}{Site Reliability Engineering Team Lead}{Crunch.io (YouGov)}{San Francisco}{CA (remote)}{
\begin{itemize}
\item Led a six person team tasked with delivering a more stable and performant back end while increasing security and improving processes, as well as working on increasing developer productivity
	\begin{itemize}
	\item Extensive project planning/management to plan for the quarter/year ahead
	\item Reported directly to the C-suite related to on-going reliability issues, and stability concerns
	\item Led 1 on 1's with my team to better understand where they wanted to go in their careers
	\item Worked extensively with the support team to understand customer complaints
	\item Helped prioritize and task firefighting responsibilities and was part of the front lines helping fix the service
	\end{itemize}
\item System administration with Ansible to deploy to AWS: EC2/Route53/S3
\item Primary security expert at Crunch helping build/improve the company's security posture, including applying security patches to Linux machines, and building security into the development lifecycle
\item Built a CI pipeline for Jenkins using docker that does per-branch builds increasing developer productivity
\item Used AWS EC2 spot instances with auto-scaling to allow Jenkins to spin up/down executors on demand
\item Documented and taught the team at Crunch about corporate security, implemented 2FA and SSO
\item Actively contributed to the source code, including work to migrate from Python 2 to Python 3
\item Open sourced software for securing against SSRF, and to help run Selenium tests
\end{itemize}
}

\cventry{2015--2016}{System Architect}{Centurylink, Inc}{Littleton}{CO}{
\begin{itemize}
\item Worked with a team to architect the next generation automation platform for video delivery
	\begin{itemize}
	\item Led effort working with vendors to build appropriate hardware configurations for the new IaaS platform
	\item Built requirements document with Canonical for the OpenStack/MaaS/JuJu platform
	\item Documented network, physical layouts, and software requirements for next generation platform
	\item Closely worked with other teams within the organisation to provide solutions that match their requirements (Platform as a Service, virtual machines, containers and others)
	\end{itemize}
\item Configured Juniper QFX/Arista switches to configure networks for Ceph/OpenStack
\item Extensively stress tested and verified Ceph cluster storage would meet project requirements
\item Automated installation of new server nodes using Metal as a Service
\item Managed and deployed JIRA, Confluence and Gitlab for internal developer tools
\end{itemize}
}

\cventry{2014--2015}{Cross Domain Engineer}{Charter Communications}{Englewood}{CO}{
\begin{itemize}
\item Deployed, installed, and configured OpenStack using Red Hat Enterprise Linux 7
\item Developed custom applications using OpenStack API's using Python and Pyramid
\item Wide range of platforms/technologies for IaaS: Linux KVM (OpenStack)/VMWare vCenter, Cisco UCS, NetApp, Cisco Nexus 5K/7K, A10 Networks/F5 load balancing
\item Leveraged Python for network automation/software defined networking for Cisco switches/routers
\item Collaborated with the network team to deploy configurations for Cisco NX-OS, IOS, IOS XR
\item Integral team member provided assistance to the deployment, architecture and operation of the new CTEC lab facility
\end{itemize}
}

\section{Interests \& Projects}
\cvitem{open source}{Core maintainer for the \href{http://www.pylonsproject.org}{Pylons Project}, specifically: \href{https://github.com/Pylons/webob}{WebOb} a very popular request/response library for Python, \href{https://github.com/pylons/waitress/}{Waitress} a pure Python HTTP server for WSGI applications, and \href{https://github.com/Pylons/pyramid}{Pyramid} a Python web framework. I also help manage pull requests, bug fixes and support other projects under the Pylons Project umbrella. I have released various other projects under open source licenses. Additional open source personal projects with on-going code commits available on \href{https://github.com/bertjwregeer}{Github/bertjwregeer}.}
\cvitem{}{Voting member of the Python Software Foundation}
\cvitem{funcptr}{Technical notebook named funcptr (short for function pointer) at \url{http://funcptr.net/}. Includes documentation for solutions to problems I have encountered, with the hope of helping others solve similar problems more easily.}

\section{Education}
\cventry{2006--2009}{Software Engineering}{University of Advancing Technology}{Tempe, AZ}{}{Bachelor of Science: Software Engineering \newline{Major: Computer Programming}}

\clearpage
\end{document}

